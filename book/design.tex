\chapter{Использование библиотек кода}

Для облегчения повторного использования кода программу принято разделять на библиотеки. Библиотека - это файл, содержащий набор модулей и классов (хотя обычно каждая библиотека содержит только один модуль или класс).

Для использования библиотеки требуется явно объявить это интерпретатору. В свою очередь, интерпретатор автоматически находит и извлекает содержимое библиотеки. Обычно используемые библиотеки объявляются в начале программы.

Поиск всех объявленных библиотек происходит в каталогах, хранящихся в глобальном массиве \verb!$LOAD_PATH ($:)!. Поиск файла выполняется, начиная с первого элемента (начиная с первого каталога).

\begin{methodlist}
  \declare{.require(path)}{\# -> bool}
  Однократное использование библиотеки. Названия библиотек сохраняются в массиве \verb!$LOAD_FEAUTURES ($”)!. Для каждой библиотеки может существовать только одно объявление. Уровень безопасности объявляемой библиотеки должен быть равен 0.

  В названии файла библиотеки расширение обычно не указывается. По умолчанию обрабатывается расширение \verb!.rb!. Если файла с таким расширением не найдено, то будет произведен поиск бинарного файла с тем же именем (например, с расширениями \verb!.so! или \verb!.dll!).
   
  Возвращается логическое значение. 

  \declare{.require_relative(path)}{\# -> bool}
  Версия метода, аналогичная предыдущему. Вызывается для поиска библиотек в базовом каталоге программы.
 
  \declare{.load( path, anonym = false )}{} 
  Многократное использование библиотеки (извлечение и выполнение кода). В имени файла должно быть указано его расширение.

  Код библиотеки может быть выполнен в теле анонимного модуля. В этом случае она не будет влиять на глобальную область видимости основной программы. 
 
  \declare{.autoload( name, path )}{\# -> nil}
  Данный метод используется для автоматизации. Поиск библиотеки выполняется только при вызове переданной константы.
 
  \declare{.autoload?(name)}{\# -> path} 
  Возвращает название библиотеки, которая будет использована при вызове переданной константы. Если такая библиотека не объявлена, то возвращается nil. 
 
  \declare{module.autoload( name, path )}{\# -> nil} 
  Данный метод используется для автоматизации. Поиск библиотеки выполняется только при вызове в теле модуля переданной константы.
 
  \declare{module.autoload?(name)}{\# -> path} 
  Возвращает название библиотеки, которая будет использована при вызове в теле модуля переданной константы. Если такая библиотека не объявлена, то возвращается nil.
\end{methodlist}