\hypertarget{apppack}{}
\chapter{Упаковка данных}

Произвольные данные могут быть представлены в виде двоичного текста. Для этого элементы массива (данные) описывают с помощью форматных строк (набора спецсимволов).

\itemtitle{Синтаксис форматных строк}
\begin{itemize}
  \item Пробелы внутри форматной строки игнорируются;

  \item Цифра после спецсимвола соответствует количеству элементов, на которые распространяется его действие;
  
  \item Символ звездочки (*) после спецсимвола распространяет его на все оставшиеся элементы;
  
  \item Спецсимволы sSiIlL могут начинаться с знака подчеркивания или восклицательного знака, означающих что размер типа данных зависит от операционной системы;

  \item Добавление символов > или < позволяет использовать старший или младшие порядки байтов соответственно (l_> или L!<).
\end{itemize}

\begin{keylist}{Целые числа:}
  
  \firstkey{C} - 8-битное целое число без знака (unsigned integer или unsigned char); 
  
  \key{c} - 8-битному целое число со знаком (signed integer или signed char); 
  
  \key{S} - 16-битное целое число без знака (uint_16t); 
  
  \key{s} - 16-битное целое число со знаком (int_16t); 
  
  \key{L} - 32-битное целое число без знака (uint_32t); 
  
  \key{l} - 32-битное целое число со знаком (int_32t); 
  
  \key{Q} - 64-битное целое число без знака (uint_64t); 
  
  \key{S_ (S!)} - целое число без знака минимального размера (unsigned short);
  
  \key{s_ (s!)} - целое число со знаком минимального размера (signed short);
  
  \key{I (I_ или I!)} - целое число без знака (unsigned integer);
  
  \key{i (i_ или i!)} - целое число со знаком (signed integer);
  
  \key{L_ (L!)} - целое число без знака максимального размера (unsigned long);
  
  \key{l_ (l!)} - целое число со знаком максимального размера (signed long);
  
  \key{N} - 32-битное целое число без знака со старшим порядком байтов (для сетей);
  
  \key{n} - 16-битное целое число без знака со старшим порядком байтов (для сетей);
  
  \key{V} - 32-битное целое число без знака с младшим порядком байтов (VAX);
  
  \key{v} - 16-битное целое число без знака с младшим порядком байтов (VAX);
  
  \key{U} - кодовая позиция UTF-8 символа;
  
  \key{w} - BER-кодированное целое число.
\end{keylist}

\begin{keylist}{Десятичные дроби:}
  
  \firstkey{D, d} - число с плавающей точкой двойной точности; 
  
  \key{F, f} - число с плавающей точкой; 
  
  \key{E} - число с плавающей точкой двойной точности, с младшим порядком байтов; 
  
  \key{e} - число с плавающей точкой с младшим порядком байтов; 
  
  \key{G} - число с плавающей точкой двойной точности, со старшим порядком байтов; 
  
  \key{g} - число с плавающей точкой со старшим порядком байтов. 
\end{keylist}

\begin{keylist}{Текст:}
  
  \firstkey{A} - произвольный двоичный текст с удаленными конечными нулями и ASCII пробелами; 
  
  \key{a} - произвольный двоичный текст; 
  
  \key{Z} - произвольный двоичный текст заканчивающемуся нулем;
  
  \key{B} - произвольный двоичный текст (MSB первый); 
  
  \key{b} - произвольный двоичный текст (LSB первый); 
  
  \key{H} - шестнадцатеричный текст (hight nibble первый); 
  
  \key{h} - шестнадцатеричный текст (low nibble первый); 
  
  \key{u} - UU-кодированный тексту; 
  
  \key{M} - MIME-кодированный тексту (RFC2045); 
  
  \key{m} - base64-кодированный текст (RFC2045, если заканчивается 0, то RFC4648); 
  
  \key{P} - указатель на контейнер (текст фиксированной длины); 
  
  \key{p} - указатель на текст, заканчивающийся нулем. 
\end{keylist}

\begin{keylist}{Остальное:}
  
  \firstkey{@} - интерпретатор пропускает указанное целым числом количество байтов; 
  
  \key{X} - интерпретатор продвигается вперед на один байт;
  
  \key{x} - интерпретатор возвращается назад на один байт.
\end{keylist}
