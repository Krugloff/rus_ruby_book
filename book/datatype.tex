\chapter{Типы данных}

\section{Логические величины}

Классы TrueClass, FalseClass и NilClass - это собственные классы объектов true, false и nil соответственно. Также существуют константы TRUE, FALSE, NIL, которые могут быть переопределены (но зачем?).

\begin{longtable}{ | * {3} { l |}}
\hline
NilClass & FalseClass & TrueClass \\ \hline
nil \& obj -> false & false \& obj -> false & true \& bool -> bool \\ \hline
nil \textasciicircum\-, bool -> bool & false \textasciicircum\-, bool -> bool & true \textasciicircum\-, bool -> !bool \\ \hline
nil | bool -> bool & false | bool -> bool & true | object -> true \\ \hline
nil.to_s -> "" & false.to_s -> "false" & true.to_s -> "true" \\ \hline
\end{longtable}

\begin{methodlist}
  \declare{nil.nil?}{\# -> true}

  \declare{nil.inspect}{\# -> "nil"}

  \declare{nil.rationalize}{\# -> (0/1)}
  \alias{to_r}

  \declare{nil.to_a}{\# -> []}

  \declare{nil.to_c}{\# -> (0+0i)}

  \declare{nil.to_f}{\# -> 0.0}

  \declare{nil.to_i}{\# -> 0}
\end{methodlist}

\section{Symbol}

Добавленные модули: Comparable

Большинство методов сначала преобразует объект в текст.

\begin{methodlist}
  \declare{::all_symbols}{\# -> array}
  Возвращает массив всех существующих экземпляров класса.
\end{methodlist}

\subsection*{Приведение типов}

\begin{methodlist}
  \declare{.to_s}{\# -> string}
  \alias{id2name}
  Преобразует значение в текст.
  \\\verb!:Ruby.to_s # -> "Ruby"!

  \declare{.inspect}{\# -> string}
  Преобразует объект в текст.
  \\\verb!:Ruby.inspect # -> ":Ruby"!

  \declare{.to_sym}{\# -> symbol}
  \alias{intern}

  \declare{.to_proc}{\# -> proc}
  Преобразует в замыкание, соответствующую методу с используемым идентификатором.
  \begin{verbatim}
  1.next # -> 2
  :next.to_proc.call(1) # -> 2\
  \end{verbatim}  
\end{methodlist}

\subsection*{Операторы}

\begin{methodlist}
  \declare{symbol <=> object}{} 
  Выполняемое выражение: \verb!symbol.to_s <=> object!.

  \declare{symbol =~ object}{} 
  Выполняемое выражение: \verb!symbol.to_s =~ object!.

  \declare{symbol[*object]}{} 
  \alias{slice}
  Выполняемое выражение: \verb!symbol.to_s[*object]! или \verb!symbol.to_s.slice(*object)!.
\end{methodlist}

\subsection*{Изменение регистра}

\begin{methodlist}
  \declare{.capitalize}{\# -> symbol} 
  Выполняемое выражение: \verb!self.to_s.capitalize.to_sym!.

  \declare{.swapcase}{\# -> symbol} 
  Выполняемое выражение: \verb!self.to_s.swapcase.to_sym!.

  \declare{.upcase}{\# -> symbol} 
  Выполняемое выражение: \verb!self.to_s.upcase.to_sym!.

  \declare{.downcase}{\# -> symbol} 
  Выполняемое выражение: \verb!self.to_s.downcase.to_sym!.
\end{methodlist}

\subsection*{Остальное}

\begin{methodlist}
  \declare{.encoding}{\# -> encoding}
  Возвращает кодировку.
  \\\verb!:Ruby.encoding # -> #<Encoding:US-ASCII>!

  \declare{.empty?}{} 
  Выполняемое выражение: \verb!self.to_s.empty?!.

  \declare{.length}{\# -> integer}
  \alias{size}
  Выполняемое выражение: \verb!self.to_s.length! или \verb!self.to_s.size!.

  \declare{.casecmp(object)}
  Выполняемое выражение: \verb!self.to_s.casecmp(object)!. 

  \declare{.next}{\# -> symbol}
  \alias{succ}
  Выполняемое выражение: \verb!self.to_s.next.to_sym! или \verb!self.to_s.succ.to_sym!
\end{methodlist}

\section{Структуры}

Добавленные модули: Enumerable 

Структуры данных - это объект, имеющий только свойства. Иногда структуры также используют при необходимости передавать множество аргументов.

Для облегчения создания структур в Ruby предоставлен класс Struct.

\begin{methodlist}
  \declare{::new( name = nil, *attribute )}{\# -> class} 
  Создание класса структуры в теле Struct и объявление переданных свойств (ссылающихся на nil). Если название класса не передано, то создается анонимный класс. Анонимный класс получит собственный идентификатор, если ему будет присвоена константа.

  Для нового класса определяется конструктор, принимающий объекты для инициализации свойств.
  \begin{verbatim}
  Struct.new "Kлюч", :объект # -> Struct::Kлюч 
  # только для примера. Никогда так больше не делайте :) 
  Struct::Kлюч.new [ 1, 2, 3 ] 
  # ->  #<struct Struct::Kлюч объект=[1, 2, 3]> \
  \end{verbatim}
\end{methodlist}
  
\subsection*{Приведение типов} 

\begin{methodlist}
  \declare{.to_s}{\# -> string} 
  \alias{inspect} 
  Возвращает информацию о структуре.
  \begin{verbatim}
  Struct::Kлюч.new( [ 1, 2, 3 ] ).to_s 
  # -> "#<struct Struct::Kлюч объект=[1, 2, 3]>"\
  \end{verbatim} 

  \declare{.to_a}{\# -> array} 
  \alias{values}
  Возвращает массив свойств структуры. 
  \\\verb!Struct::Kлюч.new( [ 1, 2, 3 ] ).to_a # -> [ [ 1, 2, 3 ] ]!
\end{methodlist}

\subsection*{Элементы структур}

Элементами структур считаются свойства. Для доступа к свойствам определены операторы \verb![]! и \verb![]=!. Они принимают идентификаторы или индексы свойств. Индексация свойств начинается с 0 и соответствует порядку, использовавшемуся при создании структуры. Индекс может быть отрицательным (индексация, начинается с -1).

Если требуемое свойство не найдено, то вызывается ошибка. 

\begin{methodlist}
  \declare{struct[attr]}{\# -> object} 
  Возвращает значение свойства структуры.
  \begin{verbatim}
  Struct::Kлюч.new( [ 1, 2, 3 ] )[ "объект" ] # -> [ 1, 2, 3 ]
  Struct::Kлюч.new( [ 1, 2, 3 ] )[ 0 ] # -> [ 1, 2, 3 ] 
  Struct::Kлюч.new( [ 1, 2, 3 ] )[ -1 ] # -> [ 1, 2, 3 ]\
  \end{verbatim}

  \declare{struct[attr]=(object)}{\# -> object}
  Изменяет значение свойства структуры.
  \begin{verbatim}
  Struct::Kлюч.new( [ 1, 2, 3 ] )[ "объект" ] = :array # -> :array
  Struct::Kлюч.new( [ 1, 2, 3 ] )[ 0 ] = :array # -> :array 
  Struct::Kлюч.new( [ 1, 2, 3 ] )[ -1 ] = :array # -> :array\
  \end{verbatim} 
 
  \declare{struct.values_at(*integer)}{\# -> array}
  Возвращает массив значений свойств с переданными индексами. При вызове без аргументов возвращается пустой массив.
  \begin{verbatim}
  Struct::Kлюч.new( [ 1, 2, 3 ] ).values_at 0, -1 
  # -> [ [ 1, 2, 3 ], [ 1, 2, 3 ] ]\
  \end{verbatim} 
\end{methodlist}

\subsection*{Итераторы}

\begin{methodlist}
  \declare{.each \{ |object| \}}{\# -> self}
  Последовательно перебирает значения свойств. 
 
  \declare{.each_pair \{ | name, object | \}}{\# -> self} 
  Последовательно перебирает идентификаторы свойств и их значения. 
\end{methodlist}

\subsection*{Остальное} 

\begin{methodlist}
  \declare{.hash}{\# -> integer} 
  Возвращает цифровой код объекта. 
  \\\verb!Struct::Kлюч.new( [ 1, 2, 3 ] ).hash # -> -764829164!

  \declare{.size}{\# -> integer} 
  \alias{length} 
  Возвращает количество свойств. 
  \\\verb!Struct::Kлюч.new( [ 1, 2, 3 ] ).size # -> 1!

  \declare{.members}{\# -> array} 
  Возвращает массив идентификаторов свойств. 
  \\\verb!Struct::Kлюч.new( [ 1, 2, 3 ] ).members # -> [ :объект ]!
\end{methodlist}