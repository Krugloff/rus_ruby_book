\chapter{Безопасность}
 
Реализация собственной системы безопасности - одна из спорных и неоднозначных особенностей в Ruby. 

Безопасность кода в Ruby обеспечивается с помощью применения различных модификаторов, запрещающих изменение значения объекта или маркирующих небезопасные данные.

Интерпретатор автоматически считает небезопасными:
\begin{enumerate}
  \item аргументы, переданные при запуске программы (элементы массива ARGV);
  \item переменные окружения (элементы массива ENV);
  \item любые данные, извлекаемые из файлов, сокетов или потоков;
  \item объекты, создаваемые на основе небезопасных данных также не считаются безопасными.
\end{enumerate}

\verb!$SAFE! ссылается на значение текущего уровня безопасности. Переменная локальна для каждого отдельного блока кода или процесса. Уровень безопасности основного процесса объявляется при запуске программы с помощью ключа \verb!-T! (по умолчанию - 0).

При нарушении ограничений безопасности вызывается ошибка \error{SecurityError}.

\section{Уровни безопасности}

Более высокие уровни безопасности наследуют ограничения более низких.

\subsection{Уровень 0}

Не предполагает ограничений.

\subsection{Уровень 1}

Запрещается:
\begin{itemize}
  \item подключать библиотеки, если их название небезопасно;
  \item выполнять произвольный код, если текст кода небезопасен;
  \item открывать файлы, если их название небезопасно;
  \item соединяться с хостом, если его название небезопасно;
  \item запускать программу с ключами \verb!-e!, \verb!-i!, \verb!-l!, \verb!-r!, \verb!-s!, \verb!-S!, \verb!-X!;
  \item выполнять методы из классов Dir, IO, File, FileTest, передавая им небезопасные аргументы;
  \item выполнять методы \method{test}, \method{eval}, \method{require}, \method{load}, \method{trap}, передавая им небезопасные аргументы.
\end{itemize}
 
Также игнорируются переменные окружения RUBYLIB и RUBYOPT и текущий каталог при поиске подключаемых библиотек.

\subsection{Уровень 2}

Запрещается выполнять методы \method{fork}, \method{syscall}, \method{exit!}, \method{Process.equid=}, \method{Process.fork}, \method{Process.setpgid}, \method{Process.setsid}, \method{Process.kill}, \method{Process.seeprioprity} передавая им небезопасные аргументы.

\subsection{Уровень 3}

Все объекты, кроме предопределенных считаются небезопасными. Вызов метода \method{.untaint} запрещается.

\subsection{Уровень 4}

Запрещается:
\begin{itemize}
  \item изменять значение безопасных объектов;
  \item подключать библиотеки с помощью \method{require} или \method{load} (разрешается в теле анонимного модуля);
  \item использовать метапрограммирование;
  \item работать с любыми процессами кроме текущего;
  \item завершать выполнение процесса;
  \item читать или записывать данные;
  \item изменять переменные окружения;
  \item вызывать методы \method{.srand} и \method{Random.srand}.
\end{itemize}

Разрешается вызывать метод \method{.eval}, передавая ему небезопасные аргументы.

\section{Модификаторы}

Модификаторы в данном случае объявляются с помощью методов.

\begin{methodlist}
  \declare{object.freeze}{\# -> self} 
  Запрещает изменение значения объекта. 
 
  \declare{object.frozen?}{} 
  Проверяет запрещено ли изменение значения объекта. 
 
  \declare{object.taint}{\# -> self} 
  Маркировка небезопасных данных. Этим модификатором отмечаются полученные внешние данные. 
 
  \declare{object.untaint}{\# -> self} 
  Маркировка безопасных данных. 
 
  \declare{object.tainted?}{} 
  Проверяет безопасно ли использование объекта. 
 
  \declare{object.untrust}{\# -> self} 
  Маркировка данных, которым программист не доверяет. Этим модификатором отмечаются данные, полученные при выполнении кода с уровнем безопасности 4. 
  \declare{object.trust}{\# -> self} 
  Маркировка данных, которым программист доверяет. 
 
  \declare{object.untrusted?}{} 
  Проверяет доверяет ли программист объекту.
\end{methodlist} 