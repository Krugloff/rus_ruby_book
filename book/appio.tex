\hypertarget{appio}{}
\chapter{Создание потоков}

Вид потока (mode) регулируется с помощью группы модификаторов.

\begin{keylist}{Модификаторы:}
  
  \firstkey{"r"} - только для чтения; 
  
  \key{"r+"} - как для чтения, так и для записи (новые данные вместо старых); 
  
  \key{"w"} - только для записи (новые данные вместо старых). При необходимости создается новый файл; 
  
  \key{"w+"} - как для чтения, так и для записи (новые данные вместо старых). При необходимости создается новый файл; 
  
  \key{"a"} - только для записи (новые данные добавляются к старым). При необходимости создается новый файл; 
  
  \key{"a+"} - как для чтения, так и для записи (новые данные добавляются к старым). При необходимости создается новый файл; 
  
  \key{"b"} - двоичный режим (может использоваться с другими). Чтение из файла выполняется в ASCII кодировке. Используется только для чтения двоичных файлов в Windows; 
  
  \key{"t"} - текстовый режим (может использоваться с другими).
\end{keylist}

При чтении или записи данных используются две кодировки: внутренняя и внешняя. Внешняя кодировка - это кодировка текста внутри потока. По умолчанию она совпадает с кодировкой ОС. Внутренняя кодировка - это кодировка для работы с полученными данными внутри программы. По умолчанию она совпадает с внешней кодировкой. Если внутренняя и внешняя кодировка отличаются, то при чтении данных выполняется автоматическое преобразование из внешней кодировки во внутреннюю, а при записи данных - из внутренней во внешнюю.

После модификатора могут быть указаны внешняя и внутренняя кодировки, разделяемые двоеточием: \verb!"w+:ascii:utf-8"!. Если внутренняя кодировка не указана, то по умолчанию используется внешняя кодировка.

Вид потока также может быть изменен с помощью дополнительного аргумента - массива ключей или набора констант из модуля File::Constants.

\begin{keylist}{Принимаемые элементы:}
  
  \firstkey{mode:} вид создаваемого потока; 
  
  \key{textmode: true} - текстовый режим; 
  
  \key{binmode: true} - двоичный режим; 
  
  \key{autoclose: true} - закрытие файла, после закрытия потока; 
  
  \key{external_encoding:} внешняя кодировка; 
  
  \key{internal_encoding:} внутренняя кодировка. Если внутренняя кодировка не указана, то по умолчанию используется внешняя кодировка; 
  
  \key{encoding:} внешняя и внутренняя кодировки в формате \verb!"external:internal"!;
\end{keylist}
   
Также принимаются элементы, влияющие на \hyperlink{appencode}{\underline{преобразование кодировок}}.